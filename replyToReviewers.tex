\documentclass[12pt]{article}

\usepackage[utf8]{inputenc}
\usepackage[top=2cm, bottom=2cm, left=2cm, right=2cm]{geometry}
\usepackage{color}
\usepackage[authoryear]{natbib}
\usepackage{amsmath}
\usepackage{subcaption}
\usepackage{siunitx}
\usepackage{setspace}
\usepackage[dvipsnames]{xcolor}

\newcounter{question}
\newcommand{\name}{00}

\renewcommand{\thefigure}{R\arabic{figure}}
\renewcommand{\thetable}{R\arabic{table}}

\newcommand{\question}[1]{\stepcounter{question} \noindent \textbf{Comment \name.\thequestion} \emph{#1}\\}

\newcommand{\answer}[1]{\noindent \textbf{Answer to \name.\thequestion} #1 \mbox{}\\}

\newcommand{\newperson}[2]{\renewcommand{\name}{#2} \setcounter{question}{0} \noindent \textbf{\large Answers to #1} \\}

\begin{document}

% Macros for commmon symbols <<<
\newcommand{\ypl}{y^{+}} %yPlus
\newcommand{\ured}{U^{*}} %reduced velocity
\newcommand{\yrms}{y^{*}_{\text{RMS}}} %root-mean-square of the normalised cylinder displacement
\newcommand{\ystr}{y^{*}} %the normalised cylinder displacement
\newcommand{\fstr}{f^{*}} %the normalised vibration frequency
\newcommand{\fn}{f_{n}} %system natural frequency
\newcommand{\fk}{f_{k}} %the coarsest grid in a grid independence study
\newcommand{\fvstr}{f^{*}_{v}} %normalise vortex shedding frequency
\newcommand{\fvk}{f_{v,\text{Karman}}} %Karman vortex shedding frequency
\newcommand{\fvkstr}{f^{*}_{v,\text{Karman}}} %normalised Karman vortex shedding frequency
\newcommand{\fcyl}{f_{\text{cyl.}}} %frequency of cylinder vibration
\newcommand{\fosc}{f_{\text{osc.}}} %frequency of cylinder oscillation
\newcommand{\fclstr}{f_{\text{Cl}}^{*}} %normalised frequency of lift coefficient
\newcommand{\flrms}{F_{\text{L,RMS}}} %root-mean-square of the lift force
\newcommand{\fl}{F_{\text{L}}} %the lift force
\newcommand{\clrms}{\text{Cl}_{\text{RMS}}} %root-mean-square of the lift coefficient
\newcommand{\cflyt}{C_{F_{L},y(t)}} %IMF component of lift that is most similar to the displacement signal in terms of temporal evolution of amplitude and frequency, differing only perhaps in phase OR the component of lift with the highest correlation to the displacement signal
\newcommand{\cflkrms}{C_{F_{L},\text{Karman},\text{RMS}}} %the Karman component of lift
\newcommand{\cflsrms}{C_{F_{L},\text{streamwise},\text{RMS}}} %the streamwise component of lift
\newcommand{\ccli}{C_{\text{Cl},i}} %the ith component of lift coefficient
\newcommand{\cclystr}{C_{\text{Cl},\ystr}} %the ith component of lift coefficient
\newcommand{\cflm}{C_{F_{L},\text{max}}} %IMF component of lift that has maximum RMS amplitude in the IMF set
\newcommand{\cyrms}{C_{y,\text{RMS}}} %the RMS of the component of lift that is most correlated with the cylinder displacement signal
\newcommand{\cclrms}{C_{\text{Cl},\text{RMS}}} %the RMS of the component of lift that is most correlated with the cylinder displacement signal (new symbol)
\newcommand{\cysys}{C_{\ystr,\ystr}} %the characteristic IMF representing the normalised cylinder displacement
\newcommand{\cclys}{C_{\text{Cl},\ystr}} %the characteristic IMF representing the lift coefficient
\newcommand{\afl}{\alpha_{F_{L}}} %ratio between two dominant IMF components of the lift

\newcommand{\angfi}{\SI{90}{\degree}} %90 deg. angle
\newcommand{\angfo}{\SI{67.5}{\degree}} %67.5 deg. angle
\newcommand{\angth}{\SI{45}{\degree}} %45 deg. angle
\newcommand{\angtw}{\SI{22.5}{\degree}} %22.5 deg. angle
\newcommand{\angon}{\SI{0}{\degree}} %0 deg. angle

\newcommand{\pfrms}{P_{\text{Fluid,RMS}}} %estimated root-mean-square of fluid power
\newcommand{\pmrms}{P_{\text{Mech.,RMS}}} %estimated root-mean-square of mechanical power
\newcommand{\etamech}{\eta_{\text{Mech.}}} %mechanical power efficiency
\newcommand{\re}{\text{Re}} %Reynolds number
\newcommand{\st}{\text{St}} %Strouhal number
\newcommand{\plag}{\theta_{y-\text{Cl}}} %Characteristic phase lag
\newcommand{\phim}{\phi_{\text{mean}}} %mean phase lag
\newcommand{\wcl}{W_{\text{cyl.}}} %mean work done by cylinder over one cycle of vibration
\newcommand{\tosc}{T_{\text{osc.}}} %mean period of cylinder oscillation
\newcommand{\meff}{m_{\text{eff.}}} %effective mass
\newcommand{\zetatot}{\zeta_{tot.}} %total damping of the system

%Macros that are shorthands in writing
\newcommand{\rms}{root-mean-square} %shorthand for root-mean-square

%Macros used in writing section on GCI study
\newcommand{\rp}{r^{p}} %refinement ratio, used in GCI study
\newcommand{\fre}{f_{\text{RE}}} %Richardson extrapolation of quantity of interest, used in GCI study

%The macros for freestream velocities
\newcommand{\uon}{\SI[per-mode=symbol]{0.1}{\metre\per\second}}
\newcommand{\utw}{\SI[per-mode=symbol]{0.2}{\metre\per\second}}
\newcommand{\uth}{\SI[per-mode=symbol]{0.3}{\metre\per\second}}
\newcommand{\ufo}{\SI[per-mode=symbol]{0.4}{\metre\per\second}}
\newcommand{\ufi}{\SI[per-mode=symbol]{0.5}{\metre\per\second}}
\newcommand{\usi}{\SI[per-mode=symbol]{0.6}{\metre\per\second}}
\newcommand{\use}{\SI[per-mode=symbol]{0.7}{\metre\per\second}}
\newcommand{\uei}{\SI[per-mode=symbol]{0.8}{\metre\per\second}}
\newcommand{\uni}{\SI[per-mode=symbol]{0.9}{\metre\per\second}}
\newcommand{\ute}{\SI[per-mode=symbol]{1.0}{\metre\per\second}}
\newcommand{\uel}{\SI[per-mode=symbol]{1.1}{\metre\per\second}}
\newcommand{\utv}{\SI[per-mode=symbol]{1.2}{\metre\per\second}}
\newcommand{\utt}{\SI[per-mode=symbol]{1.3}{\metre\per\second}}

\newcommand{\uron}{2.3}
\newcommand{\urtw}{4.5}
\newcommand{\urth}{6.8}
\newcommand{\urfo}{9.1}
\newcommand{\urfi}{11.4}
\newcommand{\ursi}{13.6}
\newcommand{\urse}{15.9}
\newcommand{\urei}{18.2}
\newcommand{\urni}{20.5}
\newcommand{\urte}{22.7}
\newcommand{\urel}{25.0}
\newcommand{\urtv}{27.3}
\newcommand{\urtt}{29.5}
% >>>
% Opening remarks <<<
\noindent {\Large \textbf{Response to the reviewers for paper OE-D20-01516: \emph{Evolution of Lift in a Pure Cruciform for Energy Harvesting}}}\\

We would like to express our gratitude to the anonymous reviewers for their time and insight. Below, we document the comments from the reviewers and our response to the comments. We indicate the changes in the original manuscript using the colour \textcolor{blue}{blue}.

\vspace{1cm}

% \newperson{Associate Editor}{AE}
% 
% \question{Question}
% 
% \answer{Answer}
% >>>
% Reviewer 1 <<<
\newperson{Reviewer 1}{R1}

\question{Why OpenFOAM is employed? If other software can be employed to solve the calculation?}

\answer{
    In our opinion, any CFD software can be employed by a proficient enough user as a means to gain important numerical insight upon a fluid mechanics project. A fluid mechanics project can be of different scales: some are targeted at an individual level, dealing with relatively idealised environments as is the case in an undergraduate class on computational fluid dynamics. However, as soon as the project is commissioned at a more advanced level, e.g. post-graduate or industrial setting, we think that the tool chosen to execute the CFD must fulfill the following criteria.

    \begin{enumerate}
        \item The algorithm required to carry out the numerical work must be built into the software. Else, the software itself must be easily extensible to include the numerical routine required. \label{enum:extensible}
        \item A high degree of automation must be possible on any aspect of the numerical work, be it meshing, running and control of the simulation, data collection and parameter variation. \label{enum:automation}
        \item Parallellisation is built in to the software. \label{enum:parallellisation}
        \item A stable software code base. \label{enum:stability}
        \item The software allows real-time collaboration with multiple authors across a common platform, allowing each iteration of the project or parts of it to be version controlled. \label{enum:collaboration}
        \item Cost-effective. \label{enum:cost}
    \end{enumerate}

    OpenFOAM, which at the most fundamental level is a collection of C++ library, is easily modified and extended (point \ref{enum:extensible}) using nothing more than a plain text editor, and then recompiled to include the required functionality. The use of extensible text editors such as VIM or Emacs greatly facilitates this process, and at zero cost.

    Any aspect of the workflow using OpenFOAM can be automated (point \ref{enum:automation}) through the use of PyFOAM, a Python library that allows fine-grain control of virtually every aspect of the simulation. In addition to automating the grid independency study and parameter variation, PyFOAM can also be used to post-process data by exploiting the NumPy Python library. Furthermore, parallellisation is built into the OpenFOAM code (point \ref{enum:parallellisation}), and the fact that the code is open source, facilitates the discovery and patching of bugs (point \ref{enum:stability}).

    Usage of the OpenFOAM software involves including and modifying configuration files, written as plain text files. As such, one can easily leverage the pre-existing tools used by software engineers such as Git, enabling real-time collaboration (point \ref{enum:collaboration}) between project members and provide much-needed version control of the whole project. We would like to point out all the tools mentioned in the previous discussion are available for free, incurring zero extra cost on the team, which can be beneficial especially for new and emerging laboratories, where procurement of hardwares is of higher priority.
}

\question{Mesh independence test is carried out using GCI method, but I think the reference can be updated. For example:[Applied Thermal Engineering. 2020;171:115090], [https://doi.org/10.1016/j.energy.2020.118690].}

\answer{We included a more recent example of the GCI study in the revised manuscript at ... following the suggestion of the reviewer.}

\question{Some figures' resolution need to be adjusted.}

\answer{We exported the figures used in the revised manuscript into the PDF instead of the PNG format used in the previous version of the manuscript for better resolution as per the suggestion of the reviewer.}

\question{Boundary conditions need to be shown.}

\answer{We included the boundary conditions in a table in the revised manuscript.}
% >>>
% Reviewer 2 <<<
\newperson{Reviewer 2}{R2}

\question{It's suggested to define FIM in line 37 of page 3 (sub-section 2.1), even though it's easy to know that FIM means flow-induced motion.}

\answer{Answer}

\question{Why chose the distance of $7.5D$ from front/back boundary to centre of cylinder (Fig 2)? And how did the boundary condition set in this study?}

\answer{Answer}

\question{It's suggested to presented or validated the time step (or nondimensionalised time step) of the simulation in this work.}

\answer{what is gi}

\question{In line 30 of page 15, why the sudden jump followed by a gradual drop and a gradual rise in $\yrms$ can be observed in this study but not in woks of\citet{Nguyen2012} nor \citet{Koide2013}? Any difference of parameters leads to the different results?}

\answer{Answer}

\question{The CFD over predict the frequency response and the value of St in low reduced velocity range (Fig 10), is this caused by the boundary condition or the size of computational domain?}

\answer{Answer}

\question{It's interested that the fluctuation exists in Fig 11(a) when $\ured = \urte$, how this exist and how the value of $\yrms$ and $\fstr$ calculated for this case (The values can be found in Fig 10)?}

\answer{Answer}

\question{Looks like that the values still going to decline in Fig 12(a). It means that the vibration here still doesn't stable?}

\answer{Answer}

\question{How it's possible that $\pfrms < \pmrms$ for a given reduced velocity (Fig 20)?}

\answer{Answer}

\question{The fluid power is possible to improve by redirect the energy from the Karman to the streamwise vortex, and how it realizes? It's suggested to clarify a possible measure for it, or try to put forward an example.}

\answer{Answer}
% \newperson{Reviewer 3}{R3}
% 
% \question{Question}
% 
% \answer{Answer}
% >>>

\bibliographystyle{cas-model2-names}
\bibliography{references}

\end{document}
